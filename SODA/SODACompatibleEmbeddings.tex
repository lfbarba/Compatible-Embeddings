\documentclass{patmorin}
\usepackage[utf8]{inputenc}
\usepackage{amsmath,amsfonts,amssymb,amsthm,graphicx,graphics}
%\setcounter{tocdepth}{3}
\usepackage{graphicx}

\usepackage{lineno}
\linenumbers
%\usepackage{algorithmic}
%\usepackage{algorithm}
\usepackage{hyperref}
\usepackage[dvipsnames]{xcolor}
\definecolor{linkblue}{named}{Blue}
\hypersetup{colorlinks=true, linkcolor=linkblue,  anchorcolor=linkblue,
citecolor=linkblue, filecolor=linkblue, menucolor=linkblue,
urlcolor=linkblue}

\setlength{\parskip}{2ex}

\usepackage[margin=2.45cm]{geometry}
%\usepackage{comment}
%\usepackage{url}
\usepackage{xspace}
%\usepackage{lineno}
%\graphicspath{{img/}} % No need to write this for every figure
%\linenumbers


%\usepackage{geometry}

\newcommand{\appendixproof}{(proof in Appendix)\xspace}
\newcommand{\etal}{\emph{et al.}}

\newtheorem{theorem}{Theorem}[section]
\newtheorem{corollary}[theorem]{Corollary}
\newtheorem{lemma}[theorem]{Lemma}
\newtheorem{proposition}[theorem]{Proposition}
\newtheorem{observation}[theorem]{Observation}
\newtheorem{problem}[theorem]{Problem}
\newtheorem{definition}[theorem]{Definition}
\newtheorem{conjecture}[theorem]{Conjecture}
\newtheorem{question}[theorem]{Question}

\DeclareMathOperator{\rank}{rank}

\newcommand{\R}{\mathbb{R}}


%%
%% Here you may place your macros using \newcommand{}{}
%%

\newcommand{\red}[1]{{\color{red} #1}}

\newcommand{\ch}[1]{\ensuremath{\textsc{ch}(#1)}}

\title{\MakeUppercase{Compatible Connectivity-Augmentation \newline of Planar Disconnected Graphs}}

%\author{Second Bellairs Workshop on Geometry and Graphs}
\author{Greg Aloupis,\thanks{Department of Computer Science, Tufts University, 
                             \email{aloupis.greg@gmail.com}}\,\,
       Luis Barba,\thanks{School of Computer Science, Carleton University
                          and Département d'Informatique, 
                          Université Libre de Bruxelles,
                          \email{lbarbafl@ulb.ac.be}}\,\,
       Paz Carmi,\thanks{Department of Computer Science,
                         Ben-Gurion University of the Negev,
                         \email{carmip@cs.bgu.ac.il}}\,\,
       Vida Dujmović,\thanks{School of Computer Science 
                             and Electrical Engineering,
                             University of Ottawa,
                             \email{vida@cs.mcgill.ca}}\,\,
       Fabrizio Frati,\thanks{Dipartimento di Ingegneria,
                              Università degli Studi Roma Tre,
                              \email{frati@dia.uniroma3.it}}\,\,
       and Pat Morin\thanks{School of Computer Science, Carleton University,
                            \email{morin@scs.carleton.ca}}}



\begin{document}

\begin{titlepage}

\maketitle
\begin{abstract}
%\setlength{\baselineskip}{16.8pt}
Motivated by applications to (graph) morphing, we consider the following
\emph{compatible connectivity-augmentation problem}: We are given a
labelled planar graph, $\mathcal{G}$, with $n$ vertices that has $r>1$
connected components, and $k>1$ isomorphic planar straight-line drawings,
$G_1,\ldots,G_k$, of $\mathcal{G}$. We wish to augment $\mathcal G$
by adding  vertices and edges to make it connected in such a way that
these vertices and edges can be added to $G_1,\ldots,G_k$ as points and
straight-line segments, respectively, to obtain $k$ planar straight-line
drawings isomorphic to the augmentation of $\mathcal G$.  We show
that adding $\Theta(nr^{1-1/k})$ edges and vertices to $\mathcal{G}$
is always sufficient and sometimes necessary to achieve this goal.
Our upper and lower-bounds hold for all $1 < r < \Omega(n)$ and $k>1$.
\end{abstract}

\end{titlepage}

%A category including the fourth, optional field follows..

\section{Introduction}
\vspace{-.1in}
Consider the following problem, which will be formalized
below.  We are given several different planar straight-line drawings (or simply drawings)
of the same disconnected graph, $\mathcal G$.
We wish to make $\mathcal G$ connected by adding vertices and edges in
such a way that these vertices and edges can also be added to the planar drawings of $\mathcal G$ while preserving planarity.  The objective is to do this while minimizing the number
of edges and vertices added. As we will show later, it is not always possible to just add edges to $\mathcal G$; sometimes additional vertices are necessary.


The motivation for this work comes from the problem of morphing
planar graph drawings, which has many applications \cite{erten.kobourov.ea:intersection,friedrich.eades:graph,gotsman.surazhsky:guaranteed,surazhsky.gotsman:controllable,surazhsky.gotsman:intrinsic} including
computer animation. Imagine an animator who wishes to animate a scene in which a character's expression
goes from neutral, to surprised, to happy. The animator can draw these
three faces, but does not want to hand-draw the 30--60 frames required
to animate the change of expression. This leads to the problem of computing a {\em morphing} (i.e., a continuous deformation) that transforms one drawing of a planar graph into
another drawing of the same planar graph while maintaining planarity
of the drawing throughout the deformation.
The morphing problem has been studied since 1944, when Cairns \cite{cairns:deformations}
showed that such a transformation always exists.  Since then, a sequence of results has shown
that such transformations can be done efficiently, so that the
vertex motion can be described concisely \cite{alamdari.angelini.ea:morphing,
angelini.dalozzo.ea:morphing,grunbaum.shephard:geometry,thomassen:deformations}.  The most recent such result
\cite{angelini.dalozzo.ea:morphing} shows that any planar drawing of
an $n$-vertex \emph{connected} planar graph can be morphed into any isomorphic
drawing using a sequence of $O(n)$ \emph{linear
morphs}, in which vertices move along linear trajectories at constant
speed.

The cited morphing algorithms require that the input graph, $\mathcal{G}$, be connected; however, in many applications this is not the case. Thus, before these morphing algorithms can be applied, $\mathcal{G}$ must be augmented to a connected graph, $\mathcal H$, and this augmentation must be compatible with all drawings of $\mathcal{G}$.  The complexity of the morph produced by a morphing algorithm depends on the size of $\mathcal H$.  Therefore, an augmentation with a small number of vertices is desirable. This motivates the theoretical question studied in the current paper.

\paragraph{Formal Problem Statement and Main Result.} 
A \emph{drawing} of a labelled graph $\mathcal{G}=(V,E)$ is a one-to-one
function $\varphi\colon V\to\R^2$.  A drawing is \emph{planar} if
(a)~for every pair of edges $uw$ and $xy$ in $E$, the open line segment
with endpoints $\varphi(u)$ and $\varphi(w)$ is disjoint from the open
line segment with endpoints $\varphi(x)$ and $\varphi(y)$ and (b)~for
every edge $uw$ and every vertex $y$, $\varphi(y)$ is not contained in
the open line segment with endpoints $\varphi(u)$ and $\varphi(w)$.
Two planar drawings, $\varphi_1$ and $\varphi_2$, of $\mathcal{G}$
are \emph{isomorphic} if there exists a continuous family of planar
drawings $\{\varphi^{(t)} \colon 0\le t\le 1\}$ of $\mathcal{G}$ such
that $\varphi^{(0)}=\varphi_1$ and $\varphi^{(1)}=\varphi_2$.\footnote{By
Cairn's result, this is equivalent to saying that the two drawings of
$G$ have the same rotation schemes, the same cycle-vertex containment
relationship, and the same outer face.}


Throughout this paper we will avoid repeatedly
referencing drawing functions like $\varphi$.
Instead, we will talk about a labelled graph $\mathcal{G}$ and $k$ isomorphic
geometric graphs $G_1,\ldots,G_k$.  This means that
each $G_i$ is the geometric graph given by the drawing of $\mathcal{G}$
with some function $\varphi_i$ and that $\varphi_1,\ldots,\varphi_k$ are
pairwise isomorphic; we will also say that $G_i$ is a {\em geometric planar embedding} of $\mathcal{G}$. When necessary, we may talk about the vertex $v$
in $G_i$ where $v$ is actually a (labelled) vertex of $\mathcal{G}$; this should
be taken to mean the vertex $\varphi_i(v)$ in $G_i$.

We are now ready to state the main problem studied in this paper.  Given $k$
geometric planar isomorphic embeddings $G_1, \ldots, G_k$ of $\mathcal
G$, a \emph{compatible augmentation}, $\mathcal H$, of $\mathcal G$ is
a supergraph of $\mathcal G$ such that (1) $\mathcal H$ is connected
and (2) there exist geometric planar isomorphic embeddings, $H_1,
\ldots, H_k$, of $\mathcal H$ such that $H_i\supset G_i$ for every
$i\in\{1,\ldots,k\}$.  In this paper, we study the following extremal question: In the worst case
(over all $n$-vertex planar graphs, $\mathcal G$, and over all sets of
$k$ geometric planar isomorphic embeddings of $\mathcal{G}$), what is the size of the smallest compatible augmentation of $\mathcal G$?
We show that, if $\mathcal{G}$ has $n$ vertices and $r$ connected components, there always exists a compatible augmentation of size $O(nr^{1-1/k})$. Furthermore, this bound is tight, as there exists a graph $\mathcal G$ with $r$ components and $k$
drawings for which any compatible augmentation has
size $\Omega(nr^{1-1/k})$.

\paragraph{Related Work.} To the best of our knowledge, there is little work on
compatible connectivity-augmentation of planar graphs, though
there is work on isomorphic triangulations of polygons.   In this setting, the goal is to compatibly triangulate the interior of two planar polygons $n$-gons $P$ and $Q$.
%the graph $\mathcal{G}$ is a cycle and one has two planar straight-line drawings, $P$ and $Q$, of $\mathcal G$. The goal is to augment $\mathcal G$ (and the two drawings $P$ and $Q$) so that $\mathcal G$ becomes a near-triangulation, and $P$ and $Q$ become (geometric) triangulations of the interiors of the polygons whose boundaries are $P$ and $Q$.
Aronov \etal\ \cite{aronov.seidel.ea:compatible} showed that this can always
be accomplished with the addition of $O(n^2)$ vertices and that
$\Omega(n^2)$ vertices are sometimes necessary.  Kranakis and Urrutia
\cite{kranakis.urrutia:isomorphic} showed that the
number of triangles required is $O(n+pq)$ where $p$ and $q$ are the
number of reflex vertices of $P$ and $Q$, respectively.


Babikov \etal\ \cite{babikov.souvaine.ea:constructing} extended the result of Aronov \etal\ to polygons with holes. This work is the most closely related to ours because it encounters (the special case $k=2$ of) our problem as a subproblem. In their setting, $\mathcal G$ is a collection of $r$ cycles, the two embeddings are such that one cycle of $\mathcal G$ contains all the others in its interior and no other pair of cycles is nested. In the first stage of their algorithm, they build a connected supergraph $\mathcal{H}'$ of $\mathcal{G}$, but their supergraph has size $\Theta(n^2)$ in the worst case.  A byproduct of our main theorem is that this step of their algorithm could be done with a graph $\mathcal{H}'$ having only $O(nr^{1/2})$ edges (but completing this graph to a triangulation may still require $\Omega(n^2)$ edges in the worst case).

Finally, several papers~\cite{aghtu-acgg-08,rw-acpgg-12,t-capsg-12} have dealt with the problem of augmenting the connectivity of any (single) geometric planar graph while adding few vertices and edges.

\noindent\textbf{Outline.} To guide the reader, we give a sketch of our upper bound proof: For each component, $\mathcal C_i$, of $\mathcal G$ we select a distinguished \emph{corner}, $a_i$, in $G_1$ (a corner is the space between two consecutive edges incident to some vertex of the outer face) and call it the \emph{attachment corner} for $\mathcal C_i$.  Notice that, since $G_1,\ldots,G_k$ are isomorphic, $a_i$ appears as a corner in each of $G_1,\ldots,G_k$. Next, for $j\in \{1,\dots,k\}$, we define a connected geometric planar graph, $G_j^*$, obtained from $G_j$ after the addition of $r-1$ edges; these edges are not, in general, edges that are going to form part of the final augmentation of $G_1,\ldots,G_k$ (in fact the edges of $G_j^*$ not in $G_j$ might be different from the edges of  $G_h^*$ not in $G_h$, if $j\neq h$). We traverse the boundary of the outer face
of $G_j^*$ to obtain a polygonal path, $\varphi_j$, of length $O(n)$ that comes
close to every corner of $G_j$.  The path $\varphi_j$ is then used to define an
integer distance $d_j(a_\ell,a_m)$ for any two corners $a_\ell$ and $a_m$. This distance includes information about the number of edges of $\varphi_j$ between $a_\ell$ and $a_m$ as well the sizes of some of the components visited while walking from $a_\ell$ to $a_m$ along $\varphi_j$.

Next, we take a leap into $k$ dimensions by using the distance
functions $d_1,\ldots,d_k$ to produce a $k$-dimensional point set
$X=\{x_1,\ldots,x_r\}$ that is contained in a hypercube of side-length $O(n)$.
This mapping has the property that, by adding a path of length $O(\|x_\ell
x_m\|)$ to $\mathcal G$, the corners $a_\ell$ and $a_m$ can be joined
in each of $G_1,\ldots,G_k$ while preserving planarity.
Now, since the point set $X$ is in $\R^k$, has $r$ points, and
is contained in a hypercube of side-length $O(n)$, a result of Steele
and Snyder \cite{steele.snyder:worst} (see also Bern and Eppstein
\cite{bern.eppstein:worst}) implies that has a spanning path of
Euclidean length $O(nr^{1-1/k})$.  This implies that $\mathcal G$ can be made
connected with a collection of $r-1$ paths, whose endpoints are the
corners $a_1,\ldots,a_r$, having total size $O(nr^{1-1/k})$, and
that each of these paths can be drawn in a planar fashion in each
of $G_1,\ldots,G_k$.
At this point, it remains to show that these $r-1$ paths
can each be drawn in each of $G_1,\ldots,G_k$ without crossing
each other.  This part of the proof involves carefully winding these
paths around the components in $G_1,\ldots,G_k$ using paths close
to the paths $\varphi_1,\ldots,\varphi_k$ defined above.
%This part of the proof resembles the first part of the proof of Babikov \etal\ \cite{babikov.souvaine.ea:constructing}, but is complicated by the fact that we have to be quite careful that the number of edges in these paths remains in $O(nr^{1-1/k})$.

The lower bound involves a sequence of paths whose vertices correspond to the vertices of
a nested set of regular $\lfloor n/r\rfloor$-gons.

The remainder of the paper is organized as follows: In
Section~\ref{section:Trivial components} we solve the special case in which $\mathcal G$ has no edges. This
case is already non-trivial and introduces some of the
main ideas used in solving the full problem, which is tackled in
Section~\ref{section:General}. Section~\ref{section:Lower bound}
proves the lower bound. Due to space constraints, some of the proofs are deferred to the Appendix.
%The paper concludes with
%Section~\ref{section:Conclusions}, which summarizes and presents
%directions for future research.


%%%%%%%%%
\section{Upper bounds for trivial components}\label{section:Trivial components}
As a warmup, we consider a (trivial) graph containing $n$ vertices and no edges.
Before constructing a compatible augmentation, we provide a subroutine
that constructs a ``short'' plane spanning path of a given ordered set  of points.

\subsection{Spanning paths of point sets}
Let $S$ be a set of $n$ points in the plane with distinct $x$-coordinates.
Given a point $v\in S$, let $\rank(v)$ denote the number of points of $S$ having smaller $x$-coordinate than $v$.
Given an arbitrary order $(v_1, v_2, \ldots, v_n)$ of the points of $S$, we show how to construct a plane path that connects them in this order and such that the number of vertices between $v_{i-1}$ and $v_{i}$ is $O(|\rank(v_i) - \rank(v_{i-1})|)$, for each $i\in \{2, \dots, n\}$.

%want to construct a path $R$ that connects them in this order  such that $$|R|  = O\left(\sum_{i=1}^{n-1} |\rank(v_i) - \rank(v_{i+1})| \right)\ .$$
This paper uses the $|\cdot|$ operator in several ways.  Namely, for a real number, $x$, we denote by $|x|$ the absolute value of $x$; for a walk, $R$, we denote by $|R|$ the number of edges traversed by $R$; also, for a (weakly) simple polygon, $P$, we denote by $|P|$ the number of edges of~$P$.

Consider a horizontal line $l$ below $S$ and let $\pi$ be the closed halfspace supported by $l$ that contains $S$.  We present an algorithm that constructs $R$ iteratively; during the $i$th iteration of the algorithm, the path is extended with $O(|\rank(v_i) - \rank(v_{i-1})|)$ vertices to include $v_i$.  For each $i\in \{1,\dots,n\}$, after the $i$th iteration of the algorithm, we maintain the invariant that $\ell$ does not intersect $R$, and we also maintain the \emph{escape invariant} which is defined as follows:
For each $j\in \{i, \dots, n\}$, there is a closed cone $\Delta_{j}$ with apex $u_{j}$ such that (1) $u_j$ lies above $v_j$ and has the same $x$-coordinate as $v_j$ ($u_j =v_j$ if $j = i$), (2) $\Delta_j$ contains $v_j$ and no other point of $S$, (3) $\Delta_j$ contains the ray originating at $v_j$ in the direction of the negative $y$-axis, (4) $\Delta_j$ does not intersect $R$, and (5) $\Delta_h$ and $\Delta_j$ are disjoint inside $\pi$, for every $h\in \{i,\dots,n\}$ with $h\neq j$.

%(2) $\Delta_j$ contains the ray originating at $v_j$ in the direction of the negative $y$-axis, (3) $\Delta_{i}$ does not intersect $R$, except at $v_{i}$; also, for every $j\in \{i+1,\dots,n\}$, there is a cone $\Delta_j$ with apex $u_j$ such that (1) $u_j$ lies above $v_j$ and has the same $x$-coordinate as $v_j$, (2) $\Delta_j$ contains $v_j$ and no other point of $S$, (3) $\Delta_j$ contains the ray originating at $v_j$ in the direction of the negative $y$-axis, (4) $\Delta_j$ does not intersect $R$, and (5) $\Delta_h$ and $\Delta_j$ are disjoint inside $\pi$, for every $h\in \{i,\dots,n\}$ with $h\neq j$.
%To construct $R$, we add the points of $S$, one by one according to the given order, while maintaining the escape invariant.

Initialize $R$ as a path that consists of the single vertex $v_1$. In order to establish the escape invariant, we define $u_1=v_1$; also, for each $j\in \{2,\dots,n\}$, we define $u_j$ as an arbitrary translation up of $v_j$; further, for each $j\in \{1,\dots,n\}$, we let $\Delta_j$ be a cone with apex on $u_j$ sufficiently narrow so that these cones do not intersect inside $\pi$; see Figure~\ref{fig:Dented Halfspace} (left).

Now assume that $R$ is a path connecting $v_1$ with $v_j$, for some $j\in \{1,\dots,n-1\}$. We extend $R$ by appending a
path that connects $v_j$ with $v_{j+1}$.

First, we translate $\Delta_{j+1}$ down until its apex $u_{j+1}$ coincides with $v_{j+1}$. Let $\pi_j$ be the closure of the set obtained from $\pi$ by removing $\Delta_h$, for every $h\in\{j,j+1,\ldots,n\}$; see Figure~\ref{fig:Dented Halfspace} (right). That is, $\pi_j$ is a halfspace with dents made by the removal of $n-j+1$ cones.
%Therefore, the number of segments on the boundary of $\pi_j$ is $O(n-j)$.
Observe that, for every pair of apexes $u_i$ and $u_h$, with $i,h\geq j$, the boundary of $\pi_j$
contains a path from $u_i$ to $u_h$ with $O(|\rank(v_i)-\rank(v_h)|)$ edges.
Because $\ell$ does not intersect $R$ and by the escape invariant, the boundary of $\pi_j$ intersects $R$ only at $v_j$. Moreover, again by the escape invariant, for each $v_i$ with $i > j$, $v_i$ lies outside of~$\pi_j$ except for $v_{j+1}$ that lies on its boundary. Because both $v_j$ and $v_{j+1}$ lie on the boundary of $\pi_j$, which does not intersect $R$ other than at $v_j$, we can connect $v_j$ with $v_{j+1}$ via a path contained in the boundary of~$\pi_j$ with length $O(|\rank(v_j) - \rank(v_{j+1})|)$. In this way, we extend $R$ to a plane path that connects $v_1$ with $v_{j+1}$.

\begin{figure}[tb]
\centering
\includegraphics[width=.98\textwidth]{img/DentedHalfspace.pdf}
\caption{\small The halfplane $\pi$ and the cones $\Delta_1,\ldots,\Delta_n$ with apexes at $u_1,\ldots,u_n$ (left);
the halfspace $\pi$ before the removal of $n-j+1$ cones (middle); and the boundary of $\pi_2$ containing a path from $v_2$ to $v_3$  (right).}
\label{fig:Dented Halfspace}
\end{figure}


After connecting $v_j$ with $v_{j+1}$, for each $h\in\{j{+}2,\ldots,n\}$, either $\Delta_h$ is disjoint from $R$, or it shares some portion of its boundary with $R$. However, the interior of $\Delta_h$ does not intersect $R$.
To preserve the escape invariant, for each $h\in\{j{+}2,\ldots,n\}$, we translate $\pi$ and $\Delta_h$ downwards by a sufficiently small amount, $\varepsilon$, and we scale $\Delta_{j+1}$ horizontally down, while keeping its apex at $v_{j+1}$. To conclude, each translated or scaled cone is contained in the previous one, $u_h$ lies above $v_h$, for each $h\in\{j+2,\ldots,n\}$, and $u_{j+1}$ coincides with $v_{j+1}$. Therefore, by choosing $\varepsilon$ sufficiently small, we maintain the escape invariant and obtain the following result.

\begin{lemma}\label{lemma:Compatible augmentation for trivial components} \appendixproof
Given an order $(v_1, \ldots, v_n)$ of the vertices of $S$, there exists a plane path $R$ that connects every point of $S$ in the given order such that the number of vertices of $R$ between $v_{i-1}$ and $v_{i}$ is $O(|\rank(v_i) - \rank(v_{i-1})|)$, for each $i\in \{2,\dots,n\}$.
\end{lemma}

\vspace{-.1in}
\subsection{Compatible drawings of point sets}\vspace{-.1in}
Recall that in this section $\mathcal G$ is a graph with $n$ trivial components.
Let $G_1, \ldots, G_k$ be $k>1$ isomorphic drawings of $\mathcal G$, i.e., $G_i$ is a sequence of $n$ points in the plane.
Assume without loss of generality that no two points of $G_i$ share the same $x$-coordinate.
Given a vertex $v$ of $\mathcal G$, let $\rank_{G_i}(v)$ denote the number of points of $G_i$ having smaller $x$-coordinate than $v$, and let $x_v = (\rank_{G_1}(v), \ldots, \rank_{G_k}(v))$ be a point in the integer grid $[0;n-1]^k$ in $\mathbb{R}^k$.  Let $X = \{x_v : v\in V(\mathcal G)\}$ and let $P$ be the shortest Hamiltonian path of $X$ using the Euclidean metric.  It is known that the length of
$P$ is $O(n^{2-1/k})$~\cite{steele.snyder:worst}.
Note that the order of the points of $P$ induces an order on the vertices of $\mathcal G$ and hence, an order on the vertices of each $G_i$.


\begin{theorem}\label{theorem:points}
For each $i\in \{1,\dots,n\}$, we can construct a path $R_i$ of length $O(n^{2-1/k})$ that connects every point of $G_i$ so that $G_i\cup R_i$ is plane. Moreover, for any distinct $i,j\in \{1,\dots,n\}$, $G_i\cup R_i$ and $G_j\cup R_j$ are~isomorphic.
\end{theorem}
\vspace{-.2in}
\begin{proof}
By relabelling, let $(v_1, \ldots, v_n)$ denote the order of the vertices of $\mathcal{G}$ induced by $P$.  The augmented graph $\mathcal{H}$ is a path that visits the vertices $v_1,\ldots,v_n$ in this order. Letting $d_j$ denote the Euclidean distance between $x_{v_j}$ and $x_{v_{j+1}}$, the path $\mathcal{H}$ includes an additional $O(d_j)$ vertices between $v_{j}$ and $v_{j+1}$.  It follows that the number of vertices
in $\mathcal{H}$ is proportional to the length of $P$, which is $O(n^{2-1/k})$.

For each $G_i$, we use Lemma~\ref{lemma:Compatible augmentation for trivial components} to draw $\mathcal{H}$ as a plane path, $R_i$,
that connects the vertices $v_1,\ldots,v_n$ in this order in the drawing $G_i$.
Since $d_j\ge |\rank_{G_i}(v_j) - \rank_{G_i}(v_{j+1})|$, the $O(d_j)$
vertices in $\mathcal{H}$ between $v_j$ and $v_{j+1}$ are enough to
draw the $O(|\rank_{G_i}(v_j) - \rank_{G_i}(v_{j+1})|)$ vertices in $R_i$
between $v_j$ and $v_{j+1}$.
Since the vertices of each $G_i$ are connected in the same order,
$G_i\cup R_i$ is isomorphic to $G_j\cup R_j$ for each $i,j\in\{1,\ldots,k\}$.
\end{proof}


\section{The general problem}\label{section:General}
In this section, we extend the result presented in
Section~\ref{section:Trivial components} to graphs with
non-trivial components.  We follow the same general scheme used in
Section~\ref{section:Trivial components} for the case of trivial
(isolated vertex) components:  We define $k$ different
orderings of the components of $\mathcal G$ and use these orderings (and
the sizes of these components) to define an $r$-point set, $X$, in $\R^k$. A
short path that visits all points in $X$ is then translated back into
a short path, $R$, that visits all components of $\mathcal G$. The path
$R$ is then
added, as a polygonal path, $R_i$, to each drawing, $G_i$, of $\mathcal G$.

Unlike the case in which
components are isolated vertices, there is no natural ordering of the
components of $G_i$, so we must define one. Also, the drawing
of path $R_i$ is considerably more complicated.  In Section~\ref{section:Trivial components}, $R_i$ is drawn incrementally, and always passes above components that are not yet included in $R_i$ and below components that are already included in $R_i$.  In this section, we redefine ``above'' and ``below''. The cost of going above or below a component depends on its size and structure.
%We begin with some careful definitions because, when a component is not an isolated vertex, we have to be considerably more specific about how the path, $R$, visits it.


\subsection{Preliminaries}\label{section:Preliminaries}
Let $C$ be a connected geometric plane graph. Let $v_0, v_1, \ldots, v_k, v_0$ be the sequence of vertices of $C$ visited by a counterclockwise
Eulerian tour along the boundary of the outer face of $C$. Note that
$v_i$ may be equal to $v_j$ for some $i\neq j$.  A vertex $v_i$
in this sequence is called a \emph{corner} of $C$.  We consider the boundary of $C$, denoted by $\partial C$, to be the
boundary of the weakly-simple polygon $(v_0, \ldots, v_k, v_0)$ whose
vertex set is the set of corners of $C$.\footnote{More formally, $\partial C$ is the boundary of the unbounded component of $\mathbb{R}^2\setminus C$, where we treat $C$ as the union of all its edges and vertices.}

Let $\varepsilon >0$. For each corner $v_i$ of $\partial C$, let $\ell_i$ be the half-line starting at $v_i$ that bisects the angle between the edges $v_{i-1}v_i$ and $v_i v_{i+1}$ in the outer face of $C$. Let $z_i$ be the point at distance $\varepsilon$ from $v_i$ along $\ell_i$. We call $z_i$ the \emph{$\varepsilon$-copy} of $v_i$. Let $\partial_\varepsilon C$ be the polygon defined by the sequence $(z_0, z_1, \ldots, z_k, z_0)$, i.e., $\partial_\varepsilon C$ is isomorphic to $\partial C$ (but not necessarily to $C$). We call $\partial_\varepsilon C$ the \emph{$\varepsilon$-fattening} of $C$.
An $\varepsilon$-fattening $\partial_\varepsilon C$ is \emph{simple} if $\partial_\varepsilon C$  is a simple polygon that contains~$C$.
Note that $\partial_\varepsilon C$ is simple, provided that $\varepsilon$ is sufficiently small. In this paper, we consider only simple $\varepsilon$-fattenings; see Figure~\ref{fig:Blowing}. Note that the (graph) distance between two corners of $\partial C$ along the boundary of $C$ is the same as the distance between their $\varepsilon$-copies along~$\partial_\varepsilon C$.

%%%%%%%%%%%%%%%
\subsection{Connected augmentations}\label{section: connected augmentations}
Let $G$ be a geometric plane graph with $r$ connected components such that each component is adjacent to the outer face.
Two vertices are {\em visible} if the open segment joining them does not intersect $G$.
Let $T_G$ be a smallest set of edges of the visibility graph of $G$ that need to be added to $G$ to make it connected.
As there are always two components containing mutually visible vertices, we can connect them and repeat recursively.  Thus, $T_G$ has $r-1$ edges. (Loosely, we can think of $T_G$ as a spanning tree of $G$'s components.) Let $G^* = G\cup T_G$.  We say that $G^*$ is a \emph{connected augmentation} of $G$; see Figure~\ref{fig:Blowing}.

\begin{figure}[h]
\centering
\includegraphics{img/Blowing.pdf}
%[width=1\textwidth]
\caption{\small A geometric graph $G$ (left); a connected augmentation, $G^*$, of $G$ (middle); and the $\varepsilon$-fattening, $\partial_\varepsilon G^*$ (right).}
\label{fig:Blowing}
\end{figure}

Let $C_1, \ldots, C_r$ be the components of $G$.
%Recall that we consider $\partial C_i$ to be the boundary of a weakly-simple polygon. Therefore, even though a vertex of $C_i$ can appear multiple times along $\partial C_i$, we consider them as different corners of $\partial C_i$.
For each $i\in \{1,\dots,r\}$, let $a_i\in C_i$ be an arbitrary corner of $\partial C_i$ (note that $a_i$ is adjacent to the outer face).
We call $a_i$ the \emph{attachment corner} of $C_i$.

Let $\varphi$ be the path on the corners of $\partial G^*$ (hence $\varphi$ is also a walk on the vertices of $\partial G^*$) obtained by splitting $\partial G^*$ at the corner $a_1$. That is, $\varphi$ is a path that visits every corner of $\partial G^*$ exactly once except for $a_1$, that is visited twice.
Given two corners $u$ and $v$ in $\partial G^*$, let $\varphi(u,v)$ denote
the unique path in $\varphi$ that connects $u$ with $v$. Let $A(u,v)$ be
the set of attachment corners of $G$ visited by $\varphi(u,v)$. Define
    $ \sigma_G(u,v) = |\varphi(u,v)| + \sum_{a_i\in A(u,v)}|\partial C_i|$,
which we call the \emph{cost} of going from $u$ to $v$. 
%Intuitively, the cost of going from $u$ to $v$ depends heavily on the number of corners visited by $\varphi(u,v)$.

\begin{lemma}\label{lemma:Contained in integer grid}\appendixproof
  If $a$ is an attachment corner of $G$, then $\sigma_G(a_1, a) < 4n$. Moreover, if $b$ is another attachment corner of $G$, then
  $\sigma_G(a, b) =  |\sigma_G(a_1, a)- \sigma_G(a_1, b)|$.
\end{lemma}
\vspace{-.1in}
\subsection{Spanning paths for connected augmentations}\label{section:Spanning paths for connected augmentations}
Let $a_1, \ldots, a_r$ be an arbitrary order of the attachment corners of $G$ (we can get the incremental indexing by relabeling the components).
Consider a plane path $R = (\rho_1, \rho_2, \ldots, \rho_t)$ that passes through all attachment corners of $G$ and through no other vertices of $G$ ($R$ may have vertices not in $G$). We say that $C_i$ {\em lies to the right of $R$} if, for each $j\in\{1,\ldots,t\}$ such that $\rho_j=a_i$ is an attachment corner of $G$, $\rho_{j-1}$ and $\rho_{j+1}$ appear as consecutive vertices when sorting the neighbors of $a_i$ in clockwise order around $a_i$ in the graph $G\cup R$; see Figure~\ref{fig:Neighborhood} (left).

We want $R$ to connect the attachment corners of $G$ in the given order, i.e., if $i < j$, then $a_i$ is visited before $a_j$ by $R$. We want to construct $R$ so that each component $C_i$ of $G$ lies to the right of~$R$.
Moreover, we want the subpath of $R$ between $a_j$ and $a_{j+1}$ to have $O(\sigma_G(a_j, a_{j+1}))$ vertices. 
We initialize $R$ with the trivial path that contains only $a_1$, and then modify $R$ iteratively, so that each new corner $a_i$ is included in $R$.
Recall that  for any given $\varepsilon >0$, $\partial_\varepsilon G^*$ denotes the $\varepsilon$-fattening of $G^*$ (see Section~\ref{section:Preliminaries}).
Let $\mu>0$ be a small constant to be specified later.
Initially, let $\varepsilon = 2\mu$ and let $\delta = \mu/2$. Let $\lambda < \mu/2^{r+1}$ be a constant sufficiently small so that $\partial_\lambda C_i \cap \partial_\lambda C_j = \emptyset$ for any distinct $i,j\in\{1,\ldots,r\}$.
Throughout, $\lambda$ remains constant while $\varepsilon$ and $\delta$ are redefined at each iteration. However, as an invariant we maintain $\lambda < \delta < \varepsilon$.

\begin{figure}[h!]
\centering
\includegraphics[width=1\textwidth]{img/Neighborhood.pdf}
\caption{\small The component $C_i$ is to the right of the path $R=(\rho_1,\ldots,\rho_t)$ (left); the $\epsilon$-fattening of $G^*$ and the ``cones'' $\Delta_1,\ldots,\Delta_r$ (middle); and a close look at the ``cone'' $\Delta_1$ (right).}
\label{fig:Neighborhood}
\end{figure}

For each $i\in \{1,\dots,r\}$, let $w_i$ be the $\varepsilon$-copy of  $a_i$.
Split $\partial_\varepsilon G^*$ at $w_1$, i.e., $\partial_\varepsilon G^*$ is a path with both endpoints equal to $w_1$.
By choosing $\varepsilon$ sufficiently small, we guarantee that $\partial_\varepsilon G^*$ is simple, i.e., $\partial_\varepsilon G^*$ is isomorphic to $\varphi$.
We say that two points in the plane are \emph{$R$-visible} if the open segment joining them does not intersect $R$.
Let $\tau >0$. For each $i\in \{1,\dots,r\}$ such that $a_i$ is not an interior point of $R$, consider the set of points $N_i\subset \partial_\varepsilon G^*$ that are at distance at most $\tau$ from $w_i$.
Let $\Delta_i$ be the convex hull of $N_i\cup \{a_i\}$, i.e., $\Delta_i$ is a ``cone'' with apex at $a_i$; see Figure~\ref{fig:Neighborhood} (middle, right). We deliberately misuse the word ``cone'' here because the ``cones'' $\Delta_1,\ldots,\Delta_r$ in this section play the same roles as the cones $\Delta_1,\ldots,\Delta_n$ in Section~\ref{section:Trivial components}.

While constructing $R$, we also maintain the \emph{escape invariant} which is defined as follows. Assume that $R$ so far connects $a_1$ with $a_i$, for some $i\in \{1,\dots,r-1\}$. Then: (1) $R$ intersects neither $\partial_\varepsilon G^*$ nor its unbounded face; (2) for each $j>i$, $R$ intersects neither the simple polygon bounded by $\partial_\delta C_j$ nor the cone $\Delta_j$; and (3) $\Delta_h\cap \Delta_j = \emptyset$, for any distinct $h,j\in\{1,\ldots,r\}$. In particular, the escape invariant implies that every point in $N_j$ (including $w_j$) is $R$-visible from $a_j$. The escape invariant holds when $R=\{a_1\}$, provided that $\tau$ is sufficiently small.


Assume that we have constructed a path $R$ that connects $a_1$ with $a_j$, for some $j\in\{1,\ldots,r-1\}$, and that the escape invariant holds.  To extend $R$, we create a new path that connects $a_j$ with $a_{j+1}$ without crossing $R$ while maintaining the escape invariant.  Recall that we consider $\partial_\varepsilon G^*$ to be a path with both endpoints on~$w_1$.

The first part of the path connecting $a_j$ with $a_{j+1}$ consists of a path connecting $a_j$ with $w_j$. If $j=1$, or if $j>1$ and $R$ together with the edge $a_j w_j$ leaves $C_j$ to its right, then connect $a_j$ with $w_j$ via a straight-line segment; since $w_j$ is $R$-visible from $a_j$, this segment does not cross $R$. Otherwise, connect $a_j$ with $\partial_\lambda C_j$ via a straight-line segment and traverse $\partial_\lambda C_j$ in clockwise order without crossing $R$ before moving to $w_j$ on $\partial_\varepsilon G^*$. In this way, we guarantee that $C_j$ lies to the right of the constructed path; see Figure~\ref{fig: Component to the right} for an illustration. Because $\lambda < \delta < \varepsilon$ and since $a_i\in V(R)$, the escape invariant is preserved.


%If $j>1$, then we need to be careful in the neighbourhood of $a_j$ as we want $R$ to have $C_j$ to its right. If the path $R$ together with the edge $a_j w_j$ leaves $C_j$ to its right, then walk from $a_j$ to $w_j$ in straight line. Because the escape invariant holds, we know that $w_j$ is $R$-visible form $a_j$ and hence, this edge does not cross $R$. If $R$ together with $a_j w_j$ does not leave $C_j$ to its right, then walk from $a_j$ to $\partial_\lambda C_j$ instead and traverse $\partial_\lambda C_j$ in clockwise order without crossing $R$ before moving to $w_j$ on $\partial_\varepsilon G^*$. In this way, we guarantee that $C_j$ lies to the right of the constructed path; see Figure~\ref{fig: Component to the right} for an illustration. Because $\lambda < \delta < \varepsilon$ and since $a_i\in V(R)$, the escape invariant is preserved.

\begin{figure}[t]
\centering
\includegraphics{img/ComponentToTheRight.pdf}
\caption{\small When extending $R$ from $a_j$ to $a_{j+1}$ we keep $C_j$ to the right of $R$.}
\label{fig: Component to the right}
\end{figure}

The path from $a_j$ to $a_{j+1}$ continues with a path from $w_j$ to $w_{j+1}$, which follows the unique path in $\partial_\varepsilon G^*$ from $w_j$ to $w_{j+1}$. However, whenever we reach an endpoint of $N_i$ for some $i\in \{1,\dots,r\}$ such that $i>j+1$, we take a detour to the other endpoint of $N_i$ while avoiding its interior so that the points in the interior of $N_i$ remain $R$-visible from $a_i$; see Figure~\ref{fig:Skip Component}. Formally, we walk from the reached endpoint of $N_i$ to $\partial_\delta C_i\setminus \Delta_i$ along the boundary of $\Delta_i$. Then, we traverse the path $\partial_\delta C_i\setminus \Delta_i$ before moving to the other endpoint of $N_i$ from the endpoint of $\partial_\delta C_i \setminus \Delta_i$. In this way, we avoid crossing the cone $\Delta_i$. Note that $R$ does not intersect the interior of the simple polygon bounded by $\partial_\delta C_i$ nor the interior of $\Delta_i$. Moreover, $R$ remains inside the simple polygon bounded by $\partial_\varepsilon G^*$.

\begin{figure}[b]
\centering
\includegraphics{img/SkipComponent.pdf}
\caption{\small The ``detour'' taken to avoid crossing the cone $\Delta_i$ (left, middle); and the narrowing of the cone $\Delta_i$ as well as the redefinition of the $\varepsilon$- and $\delta$-fattenings of $G^*$ and $C_i$, respectively.}
\label{fig:Skip Component}
\end{figure}

Once we go around $C_i$, we are back on $\partial_\varepsilon G^*$ on the other endpoint of $N_i$. In this way, we continue going towards~$w_{j+1}$ along $\partial_\varepsilon G^*$ until reaching an endpoint of $N_{j+1}$.
Once we reach an endpoint of $N_{j+1}$, we move directly from this endpoint to $a_{j+1}$.

Because $\partial_\varepsilon G^*$ is isomorphic to $\varphi$, the constructed path between $a_j$ and $a_{j+1}$ has length at most $|\varphi(a_j, a_{j+1})|$ plus the length of the boundaries of the components that we walked around. Because each component we walked around has its attachment corner on the path $\varphi(a_j, a_{j+1})$, and thus in $A(a_j, a_{j+1})$, the length of the constructed path between $a_j$ and $a_{j+1}$ is
\[ O\left(|\varphi(a_j, a_{j+1})| + \sum_{a_i\in A(a_j, a_{j+1})} |C_i|\right) = O(\sigma_G(a_j, a_{j+1})) \enspace .
\]

After reaching $a_{j+1}$, we increase $\varepsilon$ by a factor of two. Similarly, we decrease the value of $\delta$ by a factor of two. That is, after reaching $a_{j+1}$, $\varepsilon = \mu 2^{j+1}$ while $\delta = \mu/2^{j+1}$ and hence, we guarantee that $\lambda < \delta < \varepsilon$.
Also, $\partial_\varepsilon G^*$ is still simple, provided that $\mu$ is initially chosen to be sufficiently small.
Finally, we reduce $\tau$ by a factor of two and update $N_i$ and $\Delta_i$ accordingly, for each $i\in \{1,\dots,n\}$; see Figure~\ref{fig:Skip Component} (right).

Recall that for each $a_i\notin R$, $R$ intersected neither the interior of $\Delta_i$ nor the interior of the polygon bounded by $\partial_\delta C_i$. Moreover, $R$ remained within $\partial_\varepsilon G^*$.
Therefore, after increasing (\emph{resp.} reducing) $\varepsilon$ (\emph{resp.} $\delta$), we preserve the escape invariant for the next iteration of the algorithm.
We iterate until all attachment corners of $G$ are visited by $R$.

\begin{lemma}\label{lemma:Path for connected augmentations} \appendixproof
Given an arbitrary order $a_1, \ldots, a_r$ of the attachment corners of $G$, there is a path $R$ connecting all attachment corners of $G$ in the given order such that $R\cup G$ is plane, every component $C_i$ of $G$ lies to the right of $R$ when oriented from $a_1$~to~$a_r$, and the subpath of $R$ between $a_j$ and $a_{j+1}$ has $O(\sigma_G(a_j, a_{j+1}))$ vertices, for each $j\in \{1,\dots,r-1\}$.
\end{lemma}

Figure~\ref{figure:big-example} (left) illustrates the preceding algorithm on a small example.  In this example, the path from $a_1$ to $a_2$ passes by $a_4$, so $R$ detours around $C_4$ in order to preserve the escape invariant at $a_4$.  After $R$ attaches to $a_2$ and $a_3$, it winds around components $C_2$ and $C_3$, respectively, in order to ensure that these components attach to the right of $R$.


\subsection{Compatible drawings of planar graphs}
Let $\mathcal G$ be a planar graph with $n$ vertices and $r$ connected
components.  Let $G_1, \ldots, G_k$ be $k$ plane isomorphic drawings
of $\mathcal G$.  For now, we will assume that, in these drawings,
every component of $\mathcal G$ has at least one vertex incident to
the outer face.  (In the Appendix, we show that this is not a real
restriction; one can apply the same algorithm to each face of $\mathcal
G$.)  We show how to construct a compatible augmentation of $\mathcal G$
of size $O(nr^{1-1/k})$.

Let $\mathcal C_1, \ldots, \mathcal C_r$ be the connected components of $\mathcal G$.  Because $G_1,\ldots,G_k$ are isomorphic, we can select one attachment corner from each component in the drawing $G_1$, and this attachment corner also appears in each of $G_2,\ldots,G_k$. Thus, for each $j\in\{1,\ldots,r\}$, we choose an attachment corner $a_j$ of $\partial C_j$ such that $a_j$ is incident to the outer face of $C_j$.

\begin{figure}[b]
   \centering{\includegraphics{img/big-example}}
   \caption{\small An example of the algorithm for generating a spanning path that connects $a_1,\ldots,a_4$ (left); in the lower bound of Theorem~\ref{thm:lower-bound}, all
    drawings use the same set of points for vertices and segments for
    edges (middle); and the drawing of a path that joins $\mathcal{C}_i$ to
    $\mathcal{C}_j$ must travel around all the paths embedded between the
    drawing of $\mathcal{C}_i$ and the drawing of $\mathcal{C}_j$ (right).}
   \label{figure:big-example}
\end{figure}

For each $i\in \{1,\dots,k\}$, let $G_i^*$ be a connected augmentation of $G_i$, as defined in Section~\ref{section: connected augmentations}. For each $i\in\{1,\ldots,k\}$ and $j\in\{1,\ldots,r\}$, let $\rank_i(j) = \sigma_{G_i}(a_1, a_j)$. For each $j\in\{1,\ldots,r\}$, let $x_j\in \mathbb{R}^k$ be a point corresponding to the component $C_j$ such that $x_j = (\rank_1(a_j), \rank_2(a_j), \ldots, \rank_k(a_j))$. Let $X = \{x_1, \ldots, x_r\}\subset\R^k$ denote the resulting set of points. Lemma~\ref{lemma:Contained in integer grid} implies that $X$ is contained in an integer grid of side length $4n$.

Let $P$ be the shortest Euclidean Hamiltonian path of $X$. Because $X$ is contained in the $k$-dimensional integer grid of side-length $4n$ and $|X| = r$, the maximum (Euclidean) length of $P$ is $O(nr^{1-1/k})$~\cite{steele.snyder:worst}. Note that the order of the points in $P$ induces an order of the components of $\mathcal G$ and hence an order of the attachment corners of each $G_i$.

\begin{theorem}\label{theorem:main}
For each $1\leq i\leq k$, we can construct a path $R_i$ of length $O(nr^{1-1/k})$ that connects every component of $G_i$ such that $G_i\cup R_i$ is plane. Moreover, for each $1\leq i<j\leq k$, $G_i\cup R_i$ is isomorphic to $G_j\cup R_j$.
\end{theorem}
\begin{proof}
By relabelling, let $(a_1, \ldots, a_r)$ denote the order of the attachment corners of $G_i$ induced by $P$.  Letting $d_j$ denote the Euclidean distance between $x_j$ and $x_{j+1}$, we denote by $\mathcal{H}$ a path that passes through (the vertices corresponding to corners) $a_1, \ldots, a_r$ in this order, and that includes an additional $O(d_j)$ vertices between $a_{j}$ and $a_{j+1}$.  Thus, the number of vertices in $\mathcal{H}$ is proportional to the length of $P$, which is $O(nr^{1-1/k})$.

For each $G_i$, we use Lemma~\ref{lemma:Path for connected augmentations} to draw $\mathcal{H}$ as a plane path, $R_i$, that connects $a_1, \ldots, a_r$ in this order. Since $|\rank_i(j+1) - \rank_i(j)|$ represents the difference in the $i$-th coordinates of $x_j$ and $x_{j+1}$, by the triangle inequality we infer that $|\rank_i(j+1) - \rank_i(j)| \leq  d_j$. Thus, the $O(d_j)$ vertices in $\mathcal{H}$ between $a_j$ and $a_{j+1}$ are enough to
draw the $O(|\rank_i(j+1) - \rank_i(j)|)$ vertices in $R_i$ between $a_j$ and $a_{j+1}$. 

% For each $G_i$, we use Lemma~\ref{lemma:Path for connected augmentations} to construct a plane path $R_i$ that connects the attachment corners of $G_i$ in the order induced by $P$. Assume that $(a_1, \ldots, a_r)$ is the order of the attachment corners of $G_i$ induced by $P$. Lemma~\ref{lemma:Path for connected augmentations} implies that $|R_i| = O(\sum_{j=1}^{n-1} \sigma_{G_i}(a_j, a_{j+1}))$.

%Because $\sigma_{G_i}(a_j, a_{j+1}) = |\sigma_{G_i}(a_1, a_{j+1}) - \sigma_{G_i}(a_1, a_j)|$ by Lemma~\ref{lemma:Contained in integer grid} and since $\rank_i(j) = \sigma_{G_i}(a_1, a_j)$, we get that  $\sigma_{G_i}(a_j, a_{j+1}) = |\rank_i(j+1) - \rank_i(j)|$. Therefore, $|R_i|  = O\left(\sum_{j=1}^{n-1} |\rank_i(j+1) - \rank_i(j)|\right)$.

% Let $d_j$ denote the distance between $x_j$ and $x_{j+1}$ in $P$. Because $|\rank_i(j+1) - \rank_i(j)|$ represents the difference in the $i$-th coordinates of $x_j$ and $x_{j+1}$, by the triangle inequality we infer that $|\rank_i(j+1) - \rank_i(j)| \leq  d_j$ . Thus, the path $R_i$ has length $O\left(\sum_{j=1}^{n-1} |\rank_i(j+1) - \rank_i(j)|\right) = O(\sum_{j=1}^{n-1} d_j)$. Because $\sum_{j=1}^{n-1} d_j = |P| = O(nr^{1-1/k})$, the length $R_i$ is $O(nr^{1-1/k})$.

To conclude, each $R_i$ visits each component only at its attachment corner, the attachment corners of each $G_i$ are connected in the same order, and $R_i$ leaves every component to the right when oriented from $a_1$ to $a_r$. Therefore, $G_i\cup R_i$ is isomorphic to $G_j\cup R_j$ for each $1\leq i<j\leq k$.
\end{proof}

\section{Lower Bounds}\label{section:Lower bound}
\vspace{-.1in}
Our lower bounds are based on the following lemma. It says that we can find $k$ permutations of $\{1,\ldots,r\}$ such that for half the indices $i\in\{1,\ldots,r\}$, and every $j\in\{1,\ldots,r\}\setminus\{i\}$, there is a permutation in which $i$ and $j$ are at distance $\Omega(r^{1-1/k})$.

\begin{lemma}\label{lem:permutations}\appendixproof
For $k>1$ and $1\leq r\leq n$, let $t=(1/2)^{1+1/k}\cdot(r-1)^{1-1/k}$.  There exists permutations $\pi^{(1)},\ldots,\pi^{(k)}$ of $\{1,\ldots,r\}$ such that for at least half the values of $i\in\{1,\ldots,r\}$ and for every $j\in\{1,\ldots,r\}\setminus\{i\}$,
\begin{equation}
 \max\left\{\left|\pi^{(s)}_i-\pi^{(s)}_j\right|\colon s\in\{1,\ldots,k\} \right\}
 \ge t \enspace .
     \label{eq:perm}
\end{equation}
\end{lemma}

Using Lemma~\ref{lem:permutations}, we can prove a lower bound that matches the upper bound obtained in our general construction.

\begin{theorem}\label{thm:lower-bound}
  For every positive integer $n$, every $r\in\{2,\ldots,\lfloor
  n/4\rfloor\}$, and every integer $k>1$, there exists a graph $\mathcal
  G$ having $n$ vertices, $1\leq r\leq n$ connected components, and
  $k$ isomorphic drawings $G_1,\ldots,G_k$ such that any compatible
  augmentation of $\mathcal G$ has size $\Omega(nr^{1-1/k})$.
\end{theorem}

\begin{proof}
Since the lemma only claims an asymptotic result, we may assume without
loss of generality that $r$ is even and that $2r$ divides $n$.

The graph $\mathcal G$ consists of $r$ disjoint paths,
$\mathcal{C}_1,\ldots,\mathcal{C}_r$, each of length $n/r$.  Each of the
drawings $G_1$,\ldots,$G_k$ embeds the vertices of $\mathcal G$ on the
same point and edge set. The point set consists of the vertices of
$r$ nested regular $n/r$-gons, $P_1,\ldots,P_r$, each centered at the
origin and having nearly the same size. Refer to Figure~\ref{figure:big-example} (middle). More precisely, $P_1\subset
P_2\subset\cdots\subset P_r$ and the sizes are chosen so that any segment
joining two non-consecutive vertices of $P_i$ intersects the interior
of~$P_{i-1}$.
The drawings $G_1,\ldots,G_k$ are obtained from the permutations
$\pi^{(1)},\ldots,\pi^{(k)}$ given by \linebreak Lemma~\ref{lem:permutations}.
In the drawing $G_x$, the path $\mathcal C_i$ is embedded on the vertices
of $P_{\pi^{(x)}_i}$. If $y=\pi^{(x)}_i$ is even, the drawing uses
all the edges of $P_y$ except the left-most edge.  If $y$ is odd, the
drawing uses all the edges of $P_y$ except the right-most edge.

Now, without loss of generality, consider some edge-minimal compatible
augmentation $\mathcal H$ of $\mathcal G$.  For each component
$\mathcal{C}_i$ of $G$, let $T_i$ be any path in $\mathcal H$ that has
one endpoint on $\mathcal C_i$, one endpoint on some other component
$\mathcal{C}_j$, $j\neq i$, and no vertices of $\mathcal G$ in its
interior.
Now, for each of the $r/2$ indices $i\in\{1,\ldots,r\}$ that satisfy
\eqref{eq:perm}, the path $T_i$ joins a vertex of
$P_{\pi^{(s)}_i}$ to a vertex of $P_{\pi^{(s)}_j}$, $j\neq
i$, and $|\pi^{(s)}_i-\pi^{(s)}_i|\ge t$.  This path must
have length $\Omega(tn/r)$ since it has to ``go around'' the
paths between $P_{\pi^{(s)}_i}$ and $P_{\pi^{(s)}_j}$; see
Figure~\ref{figure:big-example} (right).

Thus far, we have shown that for at least $r/2$ values of
$i\in\{1,\ldots,r\}$, the component $C_i$ is the endpoint of a
path, $T_i$, of length at least $\Omega(tn/r)=\Omega(nr^{-1/k})$.
It is tempting to claim the result at this point, since
$(r/2)\cdot\Omega(nr^{-1/k})=\Omega(nr^{1-1/k})$. Unfortunately, there
is a little more work that needs to be done, since two such paths $T_i$
and $T_j$ may not be disjoint, so summing their lengths double-counts
the contribution of the shared portion.

To finish up we note that, since the augmentation $\mathcal{H}$ is minimal,
it is a tree; $\mathcal G$ contains no cycles, so any cycle in $\mathcal H$ contains an edge not in $\mathcal G$ that could be removed.  Now, observe that if we traverse the outer face of (any planar drawing of) $\mathcal H$ then we obtain a non-simple path, $P$, that traverses each edge of $\mathcal{H}$ exactly twice. If we consider the set of maximal subpaths of $P$ with no vertex of $\mathcal G$ in their interior, we obtain a set of $r$ paths, $Q_1,\ldots,Q_{r}$ and, for every component $\mathcal C_i$ of $\mathcal G$, there is a vertex of $\mathcal C_i$ that is an endpoint of at least one such path.  Therefore, from the preceding discussion, the total length of $Q_1\ldots,Q_{r}$ is $\Omega(nr^{1-1/k})$.  But since each edge of $\mathcal H$ appears at most twice in these subpaths, we conclude that $\mathcal H$ has $\Omega(nr^{1-1/k})$ edges.  Since $\mathcal H$ is a tree, it has $\Omega(nr^{1-1/k})$ vertices.
\end{proof}

\textbf{Acknowledgements.}
This work was initiated at the \emph{Second Workshop on Geometry and Graphs},
held at the Bellairs Research Institute, March 9-14, 2014.  We are
grateful to the other workshop participants for providing a stimulating
research environment.

\newpage
% \section{Summary and Conclusions}\label{section:Conclusions}
% I have nothing to write here.  Some open problems might be nice.
\bibliographystyle{plain}
\bibliography{CompatibleEmbeddings}















\end{document}
